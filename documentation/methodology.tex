\documentclass[11pt]{article}
\usepackage[utf8]{inputenc}
\usepackage[T1]{fontenc}
\usepackage{amsmath,amssymb}
\usepackage{geometry}
\geometry{margin=1in}
\usepackage[hidelinks]{hyperref}

\title{Methodology}
\author{Dey Ex Machina}
\date{\today}

\begin{document}
\maketitle

\section{Objective}
This document compares two financial strategies over the mortgage horizon:
\begin{enumerate}
 \item buying a house and paying mortgage-related cash flows (homeowner), and 
 \item renting and investing the difference in cash flows (renter). 
\end{enumerate}
 
It answers the question: "What happens to my money if, instead of buying a house, I rent and invest?" The comparison evaluates final net worth at time T for each strategy, accounting for house price growth, rent inflation, mortgage payments, down payment, and the renter's investment returns.
For the rest of the paper we call \textbf{Bob the homeowner and Alice the renter}.

\section{Inputs}


\begin{itemize}
\item $H_0$, the value of the house at the time of purchase
\item D, the mortgage down payment (Typically 20\%)
\item $r_H$, the inflation of the house
\item $C^{Bob}_t$, the cash flow paid by Bob at time t. this can be principal payment, interest, etc...
\item $C^{Alice}_t$, the rent paid by Alice at time t
\item $r_R$, the inflation Alice rent
\item $r_I$, the return on investment for Alice
\item T, the expiry of the mortgage
\end{itemize}


\section{Methodology}

\subsection{Bob - The Homeowner}

Bob networth at maturity is simply the price of his house at the end of the mortgage:

$$V_{Bob}(T) = H_0 \cdot (1+r_H)^T$$

Before maturity, Bob networth is the value of his house minus the outstanding balance.
$$V_{Bob}(t) = H_0 \cdot (1+r_H)^t - \underbrace{P^{Bob}_t}_{\text{remaining principal balance}}$$

\subsection{Alice - The Renter}
For a fair comparison, we assume spends exactly what Bob would pay for his mortgage related cash flows.

At time 0, Alice has only her down payment D as initial investment:
$$V_{Alice}(0) = D $$
At each time step, Alice networth is updated based on her cash flows and investment returns.
There are two cases to consider:
\begin{itemize}
 \item If for a given month, Bob cash flow is higher than Alice rent, Alice pays her rent and invests the remaining cash at a rate $r_I$.
 \item If for a given month, Alice rent is higher than Bob cash flow, Alice sells from her investments to make for the rent payments
\end{itemize}
$$V_{Alice}(t+1) = V_{Alice}(t)  \cdot (1+r_I)^{\frac{1}{12}}  + \Big( C^{Bob}_{t+1} - C^{Alice}_{t+1} \Big) $$

Because we assume the rent of Alice increase at a rate $r_R$:
$$V_{Alice}(t+1) = V_{Alice}(t)  \cdot (1+r_I)^{\frac{1}{12}}  + \Big( C^{Bob}_{t+1} - C^{Alice}_{0} \cdot (1+r_R)^{t+1} \Big) $$




\end{document}

